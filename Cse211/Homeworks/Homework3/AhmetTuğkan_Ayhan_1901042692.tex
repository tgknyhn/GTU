\documentclass[a4 paper]{article}
\usepackage[inner=2.0cm,outer=2.0cm,top=2.5cm,bottom=2.5cm]{geometry}
\usepackage{setspace}
\usepackage[rgb]{xcolor}
\usepackage{graphicx}
\usepackage{verbatim}
\usepackage{subcaption}
\usepackage{amsgen,amsmath,amstext,amsbsy,amsopn,tikz,amssymb}
\usepackage{fancyhdr}
\usepackage[colorlinks=true, urlcolor=blue,  linkcolor=blue, citecolor=blue]{hyperref}
\usepackage[colorinlistoftodos]{todonotes}
\usepackage{rotating}
\usepackage{booktabs}
\newcommand{\ra}[1]{\renewcommand{\arraystretch}{#1}}

\newtheorem{thm}{Theorem}[section]
\newtheorem{prop}[thm]{Proposition}
\newtheorem{lem}[thm]{Lemma}
\newtheorem{cor}[thm]{Corollary}
\newtheorem{defn}[thm]{Definition}
\newtheorem{rem}[thm]{Remark}
\numberwithin{equation}{section}

\newcommand{\homework}[6]{
   \pagestyle{myheadings}
   \thispagestyle{plain}
   \newpage
   \setcounter{page}{1}
   \noindent
   \begin{center}
   \framebox{
      \vbox{\vspace{2mm}
    \hbox to 6.28in { {\bf CSE 211:~Discrete Mathematics \hfill {\small (#2)}} }
       \vspace{6mm}
       \hbox to 6.28in { {\Large \hfill #1  \hfill} }
       \vspace{6mm}
       \hbox to 6.28in { {\it Instructor: {\rm #3} \hfill  {\rm #5} \hfill  {\rm #6}} \hfill}
       \hbox to 6.28in { {\it Assistant: #4  \hfill #6}}
      \vspace{2mm}}
   }
   \end{center}
   \markboth{#5 -- #1}{#5 -- #1}
   \vspace*{4mm}
}

\newcommand{\problem}[2]{~\\\fbox{\textbf{Problem #1}}\hfill (#2 points)\newline\newline}
\newcommand{\subproblem}[1]{~\newline\textbf{(#1)}}
\newcommand{\D}{\mathcal{D}}
\newcommand{\Hy}{\mathcal{H}}
\newcommand{\VS}{\textrm{VS}}
\newcommand{\solution}{~\newline\textbf{\textit{(Solution)}} }

\newcommand{\bbF}{\mathbb{F}}
\newcommand{\bbX}{\mathbb{X}}
\newcommand{\bI}{\mathbf{I}}
\newcommand{\bX}{\mathbf{X}}
\newcommand{\bY}{\mathbf{Y}}
\newcommand{\bepsilon}{\boldsymbol{\epsilon}}
\newcommand{\balpha}{\boldsymbol{\alpha}}
\newcommand{\bbeta}{\boldsymbol{\beta}}
\newcommand{\0}{\mathbf{0}}


\begin{document}
\homework{Homework \#3}{Due: 04/01/21}{Dr. Zafeirakis Zafeirakopoulos}{Gizem S\"ung\"u}{}{}
\textbf{Course Policy}: Read all the instructions below carefully before you start working on the assignment, and before you make a submission.
\begin{itemize}
\item It is not a group homework. Do not share your answers to anyone in any circumstance. Any cheating means at least -100 for both sides. 
\item Do not take any information from Internet.
\item No late homework will be accepted. 
\item For any questions about the homework, send an email to gizemsungu@gtu.edu.tr
\item The homeworks (both latex and pdf files in a zip file) will be
submitted into the course page of Moodle.
\item The latex, pdf and zip files of the homeworks should be saved as
"Name\_Surname\_StudentId".$\{$tex, pdf, zip$\}$.
\item If the answers of the homeworks have only calculations without any formula or any explanation -when needed- will get zero.
\item Writing the homeworks on Latex is strongly suggested. However, hand-written paper is still accepted $\textbf{IFF}$ hand writing of the student is clear and understandable to read, and the paper is well-organized. Otherwise, the assistant cannot grade the student's homework.
\end{itemize}
\problem{1: Representing Graphs}{10}
Represent the graph in Figure \ref{fig:graph1} with an adjacency matrix. Explain your representation clearly.
\begin{figure*}[h]
    \centering
	\includegraphics[scale=0.5]{graph1.png}
	\caption{The graph for Problem 1}
	\label{fig:graph1}
\end{figure*}
\solution
\begin{table}[h]
\begin{tabular}{|l|l|l|l|l|l|}
\hline
\textbf{}                        & \textbf{a}             & \textbf{b}             & \textbf{c}             & \textbf{d}             & \textbf{e}             \\ \hline
\multicolumn{1}{|l|}{\textbf{a}} & \multicolumn{1}{l|}{0} & \multicolumn{1}{l|}{1} & \multicolumn{1}{l|}{0} & \multicolumn{1}{l|}{1} & \multicolumn{1}{l|}{0} \\ \hline
\multicolumn{1}{|l|}{\textbf{b}} & \multicolumn{1}{l|}{1} & \multicolumn{1}{l|}{0} & \multicolumn{1}{l|}{1} & \multicolumn{1}{l|}{1} & \multicolumn{1}{l|}{1} \\ \hline
\multicolumn{1}{|l|}{\textbf{c}} & \multicolumn{1}{l|}{0} & \multicolumn{1}{l|}{1} & \multicolumn{1}{l|}{1} & \multicolumn{1}{l|}{0} & \multicolumn{1}{l|}{0} \\ \hline
\multicolumn{1}{|l|}{\textbf{d}} & \multicolumn{1}{l|}{1} & \multicolumn{1}{l|}{0} & \multicolumn{1}{l|}{0} & \multicolumn{1}{l|}{0} & \multicolumn{1}{l|}{1} \\ \hline
\textbf{e}                       & 0                      & 0                      & 1                      & 0                      & 1                      \\ \hline
\end{tabular}
\end{table}
\newline\newline\newline\newline
(0) a$\to$a : There is no reflexive relation for a, so we assign 0
\newline
(1) a$\to$b : There is a direct path from a to b, so we assign 1
\newline
(0) a$\to$c : There is no direct path from a to c, so we assign 0
\newline
(1) a$\to$d : There is a direct path from a to d, so we assign 1
\newline
(0) a$\to$e : There is no direct path from a to e, so we assign 0
\newline\newline
(1) b$\to$a : There is a direct path from b to a, so we assign 1
\newline
(0) b$\to$b : There is no reflexive relation for b, so we assign 0
\newline
(1) b$\to$c : There is a direct path from b to c, so we assign 1
\newline
(1) b$\to$d : There is a direct path from b to d, so we assign 1
\newline
(1) b$\to$e : There is a direct path from b to e, so we assign 1
\newline\newline
(0) c$\to$a : There is no direct path from c to a, so we assign 0
\newline
(1) c$\to$b : There is a direct path from c to b, so we assign 1
\newline
(1) c$\to$c : There is a reflexive relation for c, so we assign 1
\newline
(0) c$\to$d : There is no direct path from c to d, so we assign 0
\newline
(0) c$\to$e : There is no direct path from c to e, so we assign 0
\newline\newline
(1) d$\to$a : There is a direct path from d to a, so we assign 1
\newline
(0) d$\to$b : There is no direct path from d to b, so we assign 0
\newline
(0) d$\to$c : There is no direct path from d to c, so we assign 0
\newline
(0) d$\to$d : There is no reflexive relation for d, so we assign 0
\newline
(1) d$\to$e : There is a direct path from d to e, so we assign 1
\newline\newline
(0) e$\to$a : There is no direct path from e to a, so we assign 0
\newline
(0) e$\to$b : There is no direct path from e to b, so we assign 0
\newline
(1) e$\to$c : There is a direct path from e to c, so we assign 1
\newline
(0) e$\to$d : There is no direct path from e to d, so we assign 0
\newline
(1) e$\to$e : There is a reflexive relation for e, so we assign 1
\newline\newline
\problem{2: Hamilton Circuits}{10+10+10=30}
Determine whether there is a Hamilton circuit for each given graph (See Figure \ref{fig:G1a}, Figure \ref{fig:G1b}, Figure \ref{fig:G1c} ). If the graph has a Hamilton circuit, show the path with its vertices which gives a Hamilton circuit. If it does not, explain why no Hamilton circuit exists. \newline
\begin{figure*}[h]
	\centering
	\begin{subfigure}[h]{0.5\textwidth}
		\centering
		\includegraphics[height=1.5in]{circuit-a.png}
		\caption{The graph $G_1$}
		\label{fig:G1a}
	\end{subfigure}%
	\begin{subfigure}[h]{0.5\textwidth}
		\centering
		\includegraphics[height=1.5in]{circuit-b.png}
		\caption{The graph $G_2$}
		\label{fig:G1b}
	\end{subfigure}
	\begin{subfigure}[h]{0.5\textwidth}
		\centering
		\includegraphics[height=1.2in]{circuit-c.png}
		\caption{The graph $G_3$}
		\label{fig:G1c}
	\end{subfigure}
	\caption{The graphs to find Hamilton circuits for Problem 1}
\end{figure*}

\subproblem{a} \solution\\
\newline
To be able to say this graph is a Hamilton circuit, first we must have a Hamilton path but we can't write a path which covers all the vertices without touching any vertices twice. The possible path should be something like a-b-c-d-e-f-g-h-i-j-k-l-m-n-o-p-q-a(this can't happen) which contains 18 vertices. But if we try to complete the path with the least number of path (for ex:  a-b-c-h-g-f-m-l-q-k-j-p-o-i-o-n-o-d-e-d-a) we write total of 21 vertices and since we write some vertices(which non 'a') more than once it isn't a Hamilton path, hence this graph is not a Hamilton circuit.
\newline
\subproblem{b} \solution\\
\newline
Like graph G1, this one is not a Hamilton circuit either. For this graph, the possible Hamilton path should be something like a-b-c-d-e-f-g-a(not possible) which contains 8 vertices with using vertex 'a' twice. But if we try to write a Hamilton path(which is also a Hamilton circuit) we see something like this: e-c-a-b-g-b-d-f-d-c-e , and this is not a Hamilton path since we used some vertices and same paths more than once. Hence this graph is not a Hamilton circuit
\newline
\subproblem{c} \solution\\
\newline
Unlike Graph G1 and G2, this one is a Hamilton circuit. A correct Hamilton path for this graph should have 10 vertices and only the first vertex must be used twice also must be same vertex with last one. An example of Hamilton circuit path of this graph is : a-b-c-f-i-h-g-d-e-a (10 vertices)
\newline
\newpage
\problem{3: Applications on Graphs}{20}
Schedule the final exams for Math 101, Math 243, CSE 333, CSE 346, CSE 101, CSE 102, CSE 273, and CSE 211, using the fewest number of different time slots, if there are no students who are taking:
\begin{itemize}
	\item both Math 101 and CS 211,
	\item both Math 243 and CS 211,
	\item both CSE 346 and CSE 101,
	\item both CSE 346 and CSE 102,
	\item both Math 101 and Math 243,
	\item both Math 101 and CSE 333,
	\item both CSE 333 and CSE 346
\end{itemize}
but there are students in every other pair of courses together for this semester.\\ 
$\textbf{Note:}$ Assume that you have only one classroom.\\ \\
$\textit{Hint 1: Solve the problem with respect to your problem session notes.}$\\
$\textit{Hint 2: \hyperlink{https://www.draw.io/}{Check the website}}$
\newline
\solution
\newline - The fewest number of different time slot is 5. 
\newline - 5 different color is used in graph.
\newline - 3 Pink - 2 Green - 1 Purple - 1 Red - 1 Blue
\newline - First, I put pink to the Math101. Then, I put different color to every relation of this course. After that, i switched to Cse333 and placed green since this course has no relation with cse346. Finally, I placed pink to Math243 and Cse211 since they have no relation with Math101.
\begin{figure}[htbp]
\centerline{\includegraphics[scale=.5]{schedule.png}}
\caption{Colored graph of Problem 3}
\label{fig}
\end{figure}
\newpage
\problem{4: Applications for Hasse Diagram of Relations}{40}
Remember the Problem 3 in Homework 2. 
\newline
\newline
Write an algorithm to draw Hasse diagram of the given relations in "input.txt" which is given for HW2.
\newline
\newline
Your code should meet the following requirements, standards and accomplish the given tasks.


\begin{itemize}
	\item Read the relations from the text file "input.txt". You can use your code from HW2 if you implemented to read the file. If you didn't implement it, please check HW2 to learn how to read the relations from the file. 
	\item Determine each relation in "input.txt" whether it is reflexive, symmetric, anti-symmetric and transitive with your algorithm from HW2.
	\item In order to draw Hasse diagram, each relation must be POSET. Hence, the relation obeys the following rules:
	
	\begin{itemize}
		\item Reflexivity
		\item Anti-symmetric
		\item Transitivity 
	\end{itemize}
	If the relation is not a POSET, your algorithm is responsible to CONVERT it to POSET. 
	\begin{itemize}
		\item If the relation is not reflexive, add new pairs to make the relation reflexive.
		\item If the relation is symmetric, remove some pairs which make the relation symmetric. For instance, if the relation has (a, b) and (b, a), remove one of them randomly.
		\item If the relation is not transitive, add new pairs which would make the relation transitive. 
	\end{itemize}
	\item After the relation becomes POSET, your algorithm should obtain Hasse diagram of the relation and write the diagram with the following format. 
	\begin{itemize}
		\item An example of the output format is given in "exampleoutput.txt". The file has the result of the first relation in "input.txt".
		\item In "output.txt", each new Hasse diagram starts with "n".
		\item The relation is (a, a), (a, b), (a, e), (b, b), (b, e), (c, c), (c, d), (d, d), (e, e)
		\item The relation is already a POSET so we don't need to add or remove any pairs.
		\item After "n", write the POSET in the next line as it is shown in "exampleoutput.txt".
		\item Since the relation is POSET, it becomes (a, b), (b, e), (c, d) after removing reflexive and transitive pairs. 
		\item The following lines give each pair of Hasse diagram.
	\end{itemize}	
	\item You can implement your algorithm in Python, Java, C or C++.
	\item $\textbf{Important: }$ Put comments almost for each line of your code to describe what the line is going to do. 
	\item You should put your source code file (file name is problem1.$\{.c, .java, .py, .cpp\}$) and output.txt into your homework zip file (check Course Policy).
\end{itemize}



\end{document} 
